%% Based on a TeXnicCenter-Template by Gyorgy SZEIDL.
%%%%%%%%%%%%%%%%%%%%%%%%%%%%%%%%%%%%%%%%%%%%%%%%%%%%%%%%%%%%%

%----------------------------------------------------------
%
\documentclass[titlepage,leqno]{beamer}%
\usetheme{Antibes}%
\usecolortheme{rose}%
\setbeamerfont{footnote}{size=\tiny}
%
%----------------------------------------------------------
% This is a sample document for the LaTeX Slides Class
% Class options
%       --  Body text point size (normalsize) is 27 (default)
%           and can not be adjusted to any other value.
%       --  Paper size:  letterpaper (8.5x11 inch, default)
%                        a4paper, a5paper, b5paper,
%                        legalpaper, executivepaper
%       --  Orientation (portrait is the default):
%                        landscape
%       --  Quality:     final(default), draft
%       --  Title page:  titlepage, notitlepage
%       --  Columns:     onecolumn (default), [twocolumn is not avalible]
%       --  Equation numbering (equation numbers on the right is the default)
%                        leqno (equation numbers on the left)
%       --  Displayed equations (centered is the default)
%                    fleqn (flush left)
%
%  \documentclass[a4paper,fleqn]{slides}
%
%  The slides are separated from each other by the slide
%  environment, see below:
%
%%%
% Some helpful information:
%
% You can typeset \emph{Emphasized text}.
%
% You can also typeset  \textbf{Bold}, \textit{Italics},
% \textsl{Slanted} and \texttt{Typewriter} text. Roman fonts are not
% available.
%
% Point size can be changed by making use of the {\tiny tiny},
% {\scriptsize scriptsize}, {\footnotesize footnotesize}, {\small
% small}, {\normalsize normalsize}, {\large large}, {\Large Large},
% {\LARGE LARGE}, {\huge huge} and {\Huge Huge} commands.
%
% The numbered equation
% \begin{equation}
% u_{tt}-\Delta u+u^{5}+u\left|  u\right|  ^{p-2}=0\text{ in }\mathbf{R}%
% ^{3}\times\left[  0,\infty\right[  .\label{eqn1}%
% \end{equation}
% is automatically numbered as equation \ref{eqn1}.
%%%

\usepackage{amsmath}%
\usepackage{amsfonts}%
\usepackage{amssymb}%
\usepackage{graphicx}%
\usepackage{adjustbox}%
\usepackage[utf8]{inputenc} % allow utf-8 input
%\usepackage[T1]{fontenc}    % use 8-bit T1 fonts
\usepackage{hyperref}       % hyperlinks
\usepackage{url}            % simple URL typesetting
\usepackage{booktabs}       % professional-quality tables
\usepackage{nicefrac}       % compact symbols for 1/2, etc.
\usepackage{microtype}      % microtypography
\usepackage{verbatim}

%%  \usepackage{bibentry} 
%%  \usepackage{natbib}
%------------------------------------------------------------------
\hfuzz5pt % Don't bother to report overfull boxes < 5pt
\pagestyle{plain}
%%% --------------------------------------------------------------
\begin{document}
\title{MINST Kaggle Digit Recognizer: Contrasting the Random Forest and MLP Neural Network}
\author{%
   Andrew Osborne \hspace{2.75cm} Josh Price\\
   \texttt{amo004@uark.edu} \hspace{1.75cm} \texttt{jdp024@uark.edu} \\
   \vspace{4mm}
   April Walker \\
   \texttt{adw027@uark.edu}
}
\institute{%
   University of Arkansas, \\
   Fayetteville, AR, 72701, USA
	}

\date{4/23/2019}
\maketitle
%-----------------------------------------------------------------
\begin{frame}{Presentation Outline}
	\begin{itemize}
		\item The MINST Dataset
		\item GridSearchCV
		\item Random Forest
		\begin{itemize}
			\item Implementation
			\item Results
		\end{itemize}
		\item Multi-Layer Perception Classifier
		\begin{itemize}
			\item Contrasting Gradient Descent Algorithms
			\item Implementation
			\item Results
		\end{itemize}
		\item{Conclusions}
	\end{itemize}
\end{frame}
% ----------------------------------------------------------------
\begin{frame}{The MINST Dataset}
%The MNIST digit database is a classic starting point for individuals wanting to learn the basics of computer vision and get hands on experience with machine learning algorithms. The dataset is composed of 70,000 28x28 pixel grey-scale images of handwritten numbers between zero and nine [6]. The pre-flattened Kaggle dataset provides you with 42,000 training examples and 28,000 testing examples. Each 28x28 pixel image is represented by a 784 length vector with each element containing some integer between 0 and 255 representing the lightness or darkness of that pixel. Additionally, the training dataset provides the correct labels for each image. These images can be reshaped and rendered from the provided dataset as shown in Figure 1. 
\begin{figure}[h]
  \centering
    \begin{minipage}[t]{0.3\textwidth}
        \includegraphics[width=\textwidth]{img786render.png}
    \end{minipage}
    \begin{minipage}[t]{0.3\textwidth}
        \includegraphics[width=\textwidth]{img98render.png}
    \end{minipage}
    \begin{minipage}[t]{0.3\textwidth}
        \includegraphics[width=\textwidth]{img6render.png}
    \end{minipage}
  \caption{Image Renderings from the MNIST Dataset}
\end{figure}
\end{frame}
%-----------------------------------------------------------------
\begin{frame}{The MINST Dataset}
\begin{itemize}
	\item 70,000 $28\times 28$ pixel grey-scale images of handwritten numbers, 0 through 9.
	\item Each image is represented by a vector of length 784, with each element taking a value between 0 and 255 to represent lightness/darkness of the pixel
	\item Pre-flattened Kaggle dataset
	\begin{itemize}
		\item 42,000 training examples
		\item 28,000 testing examples
	\end{itemize}
\end{itemize}
%The MNIST digit database is a classic starting point for individuals wanting to learn the basics of computer vision and get hands on experience with machine learning algorithms. The dataset is composed of 70,000 28x28 pixel grey-scale images of handwritten numbers between zero and nine [6]. The pre-flattened Kaggle dataset provides you with 42,000 training examples and 28,000 testing examples. Each 28x28 pixel image is represented by a 784 length vector with each element containing some integer between 0 and 255 representing the lightness or darkness of that pixel. Additionally, the training dataset provides the correct labels for each image. These images can be reshaped and rendered from the provided dataset as shown in Figure 1. 
\end{frame}
%-----------------------------------------------------------------
\begin{frame}[fragile]{GridSearchCV}
\begingroup
\small
For both machine learning algorithms, training and cross-validation was done using \verb+sklearn+'s \verb+model_selection.GridSearchCV+.

\vspace{3.75mm}

The grid search takes in a grid of the hyper-parameters the user wishes to contrast and exhaustively considers all possible parameter combinations.

\vspace{3.75mm}

\verb+GridSearchCV+ partitions the data into $k$ sections then trains the data on $k-1$ of the sections, leaving the last partition as a pseudo-test dataset.

\vspace{3.75mm}

The mean cross-validation score (CVS) is the averaged prediction rate over all $k$ segments of the training data, and the hyper-parameter combination with the best mean CVS is chosen [4]. 
\endgroup
\end{frame}
%-----------------------------------------------------------------
\begin{frame}[fragile]{Random Forest}

Each tree is formed from a bootstrap sample (random sampling with replacement) then grown very similarly to a decision tree. 

\vspace{3.75mm}

Our random forest is a ``highly random'' random forest which splits data with no discretion to promote model volatility.

\vspace{3.75mm}

\verb+Sklearn+'s \verb+RandomForestClassifier+ combines the probabilistic prediction of each tree, as this has been shown to preform better than the traditional approach of picking a classification based on a majority vote [4,5].

\end{frame}
%-----------------------------------------------------------------
\begin{frame}{Random Forest}

These models are generally very fast to train, but depending on the complexity, can suffer from slow run-time performance. 

\vspace{3.75mm}

Generally, overfitting can be negated by adding more trees to your model [2]. 

\vspace{3.75mm}

This approach eventually results in diminishing returns, and for high-dimensional problems such as digit recognition, is especially impractical [5].

\end{frame}
%-----------------------------------------------------------------
\begin{frame}[fragile]{Random Forest}
\small
\textbf{Implementation}

Data normalized to values in [0,1] using \verb+sklearn+'s \verb+preprocessing.minmax_scale+. 

\vspace{3.75mm}

In order to improve our model's predictive power, we contrasted the results from our hyper-parameters \verb+n_estimators+ (number of trees) and  \verb+min_samples_split+ (minimum number of leafs required to split a node). 

\vspace{3.75mm}

Specifically we allowed 10, 50, 100, and 300 trees to be developed and 2, 4, 8, and 16 as the minimum number of leaves. The later can be thought of as a measure of complexity, with lower minimum resulting in higher complexity. 

\vspace{3.75mm}

Once the code was prepared, training took approximately 45 seconds. 

\end{frame}
%-----------------------------------------------------------------
\begin{frame}{Random Forest}

\begin{figure}[h]
    \centering
    \begin{minipage}[t]{0.45\textwidth}
        \centering
        \textbf{(A)}
        \includegraphics[width=\textwidth]{CVscoreVsLeafSize.png}
    \end{minipage}
    \begin{minipage}[t]{0.45\textwidth}
        \centering
        \textbf{(B)}
        \includegraphics[width=\textwidth]{CVtestVsLeafSize.png}
    \end{minipage}
\caption{\footnotesize
(A,B) Cross Validation of Hyper-parameters for (A) training data and (B) testing data by contrasting mean cross validation score and $\log_2(minleafsize)$ giving the minimum leaf threshold as a measure of complexity. The scores for varying numbers of trees are shown.}
\end{figure}

\end{frame}
%-----------------------------------------------------------------
\begin{frame}[fragile]{Random Forest}
\textbf{Cross-Validation and Results}

The $\log_2(\text{minleafsize})$ which inversely measures the complexity suggests the more complex models preform outstandingly against the training data but have little impact on the testing data, a symptom of overfitting. 

\vspace{3.75mm}

Adding trees has a similar positive affect on both datasets, however, also decreases exponentially as more trees are added. These results suggest that our model suffers from extreme levels of overfitting.

\vspace{3.75mm}

Our Kaggle submission of 300 trees and a minimum leaf number of 2 gave a prediction rate of 96.6\%

\end{frame}
%-----------------------------------------------------------------
\begin{frame}{Multi-Layer Perceptron Classifier}
\textbf{Contrasting Gradient Descent Algorithms}

For our MLP, we contrast stochastic gradient descent (SGD) and Adaptive Moment Estimation gradient descent (Adam GD), developed by Kingma and Lei Ba [3]. 

\vspace{3.75mm}

\begin{itemize}
	\item SGD approximates the gradient by considering a single training example at a time.
	\item Adam GD is an optimized SGD which computes adaptive learning rates for each parameter.
\end{itemize}

\end{frame}
%-----------------------------------------------------------------
\begin{frame}{Multi-Layer Perceptron Classifier}
\tiny
\textbf{Adam Parameter Estimation}

\begin{itemize}
	\item Exponentially decaying averages of the past gradient $\nabla_\theta f_t(\theta_{t-1})$ and 
	square of the gradient $\nabla_\theta f_t(\theta_{t-1})^2$ are stored as estimates of the first 
	and second moment of the gradient (mean $m_t$ and variance $v_t$ respectively).
	\item The exponential decay rates $\beta_1$ and $\beta_2$ are initialized (generally near 1)\footnote{The 
	developers recommended default is $\beta_1 = 0.9$ and $\beta_2 = 0.999$. [3]} such that the moments can be 
	more specifically calculated using the following:
	\begin{align*}
	m_t \leftarrow& \beta_1 \cdot m_{t-1} + (1-\beta_1)\cdot \nabla_\theta f_t(\theta_{t-1})\\
	v_t \leftarrow& \beta_2 \cdot v_{t-1} + (1-\beta_2)\cdot \nabla_\theta f_t(\theta_{t-1})^2 
	\end{align*}
	\item With the initial $m_0$ and $v_0$ set to 0. In order to correct for bias, the following bias-corrected 
	moments are then computed:
	\begin{align*}
	\hat{m_t} \leftarrow& m_t/(1-\beta_1^t)\\
	\hat{v_t} \leftarrow& v_t/(1-\beta_2^t)
	\end{align*}
	\item With $\beta_1^t$ and $\beta_2^t$ indicating $\beta_1$ and $\beta_2$ to the power of $t$. 
	Given some $\alpha$ and $\epsilon$ as regularization parameters, the parameters $\theta_t$ are then 
	updated using the following:
	\begin{align*}
	\theta_t \leftarrow& \theta_{t-1} - \alpha \cdot \hat{m_t}/(\sqrt{\hat{v_t}+\epsilon})
	\end{align*}
	\item This process is repeated until $\theta$ converges [3].
\end{itemize}

\end{frame}
%-----------------------------------------------------------------
\begin{frame}{Multi-Layer Perceptron Classifier}
\scriptsize
\textbf{Implementation}

In addition to contrasting GD algorithms we also varied:
\begin{itemize}
	\item the number of hidden nodes between 128, 256, and 512
	\item the number of hidden layers either 1 or 2
	\item $\alpha$ at 0.5, 0.1, 0.001, and 0.0001
\end{itemize}
\medskip
Adam GD regularization parameters were initialized at the developer's (Kingma and Lei Ba) recommended defaults:
\begin{itemize}
	\item $\epsilon = 10^{-8}$ 
	\item $\beta_1=0.9$ 
	\item $\beta_2=0.999$ [3]
\end{itemize}
\medskip
The learning rate was not varied, instead an adaptive rate was chosen which divides the current learning rate by 5 with a starting value of $0.001$ [4]. 

\medskip

Once the code was prepared, training took approximately 46 minutes per model. Total training time took approximately 20 hours.

\end{frame}
%-----------------------------------------------------------------
\begin{frame}[fragile]{Multi-Layer Perceptron Classifier}
\scriptsize
\textbf{Cross-Validation and Results}
\begin{itemize}
	\item Training and cross-validation was done using \verb+model_selection.GridSearchCV+.
	\item The measure of complexity is given by $-\log_{10}(\alpha)$, meaning lower $\alpha$ values 
	correspond to higher complexity.
\end{itemize}
\medskip
As expected, Adam gradient descent was the clear winner across the board. 

Note:
\begin{itemize}
	\item Smaller $\alpha$ values (that is the larger $-\log_{10}(\alpha)$ values) almost consistently 
	improves the prediction rate in (A), but for (B) in some cases causes severe prediction penalties.
	\item The most obvious example of this is at 1 HL and 256 nodes in (B). 
\end{itemize} 
\medskip

Our "winning" model with 1 HL and 512 nodes preformed rather consistently between training and testing datasets.

\medskip

The Kaggle submission of this model gave a prediction rate of 97.90\%.

\end{frame}
%-----------------------------------------------------------------
\begin{frame}{Multi-Layer Perceptron Classifier}

\begin{figure}[h]
    \centering
    \begin{minipage}[t]{0.45\textwidth}
        \centering
        \textbf{(A)}
        \includegraphics[width=\textwidth]{TrainscoreVsAlpha.png}
    \end{minipage}
    \begin{minipage}[t]{0.45\textwidth}
        \centering
        \textbf{(B)}
        \includegraphics[width=\textwidth]{CVscoreVsAlpha.png}
    \end{minipage}
\caption{\scriptsize
(A,B) Cross Validation of Hyper-parameters for (A) training data and (B) testing data by contrasting the mean CVS in (A) and the prediction accuracy in (B) with $-\log_{10}(\alpha)$ giving a measure of complexity. For both (A) and (B), the left colomn gives the results for SGD, and the right colmn gives the results for Adam GD.}
\end{figure}

\end{frame}
%-----------------------------------------------------------------
\begin{frame}{Conclusions}

As expected, our MLP was better suited for handwritten digit recognition. 

\vspace{3.75mm}

It is possible that data processing to reduce dimensionality would help the RF preform better, however computer vision problems are still better left in the hands of NN's and similar models. 

\vspace{3.75mm}

One advantage of the RF is it's training time was approximately $1/60^{th}$ of our MLP's, however adding additional trees would begin to bridge this gap and further improve it's classification accuracy. 

\end{frame}
%-----------------------------------------------------------------
\begin{frame}[fragile]{References}

\tiny

[1] Bernard, S., Adam, S., \& Heutte, L. (2007). \textit{Using Random Forests for Handwritten Digit Recognition}. Ninth International Conference on Document Analysis and Recognition (ICDAR 2007) Vol 2. doi:10.1109/icdar.2007.4377074

\vspace{3mm}

[2] Hastie, T., Tibshirani, R., \& Friedman, J. H. (2017). The Elements of Statistical Learning: Data Mining, Inference, and Prediction. New York, NY: Springer.

\vspace{3mm}

[3] Kingma, D. P., \& Lei Ba, J. (2015). 
\textit{Adam: A Method for Stochastic Optimization}. Conference Paper at ICLR. Retrieved from https://arxiv.org/abs/1412.6980.

\vspace{3mm}

[4] Pedregosa \textit{et al} (2011) \textit{Scikit-learn: Machine Learning in Python}. JMLR 12, pp. 2825-2830.

\vspace{3mm}

[5] Robnik-Šikonja, M. (2004). \textit{Improving Random Forests.} Machine Learning: ECML 2004 Proceedings, Springer, Berlin, 359-370.

\vspace{3mm}

[6] Y. LeCun \textit{et al} (1995) \textit{Learning Algorithms For Classification: A Comparison On Handwritten Digit Recognition}, in Oh, J. H. and Kwon, C. and Cho, S. (Eds), Neural Networks: The Statistical Mechanics Perspective, 261-276, World Scientific.

\end{frame}
%-----------------------------------------------------------------
\begin{frame}
	\begin{center}
	\begingroup
	\huge
	Thank You!\\
	\endgroup
	\end{center}
\end{frame}
%-----------------------------------------------------------------
\end{document}
